\documentclass{article}

%Symbols
\usepackage{recycle}

%Margins
\addtolength{\voffset}{-0.5cm}
\addtolength{\hoffset}{-0.5cm}
\addtolength{\textwidth}{1cm}
\addtolength{\textheight}{1cm}

%Header-Footer
\usepackage{fancyhdr}
%Header Info
\lhead{Profesor: C\'esar Hern\'andez Cruz \\
       Ayudante: I\~naki Cornejo de la Mora}
\rhead{Gr\'aficas y Juegos 2025-2}
%Footer Info
\rfoot{\recycle}
\cfoot{\vspace{-0.8cm}?`Realmente necesitas imprimir esta hoja?}
\lfoot{\recycle}
\pagenumbering{gobble}
\footskip = 50pt
\renewcommand{\headrulewidth}{1pt}
\setlength{\headheight}{22.54448pt}

\newcommand{\set}[1]{\left\{ #1 \right\}}

\pagestyle{fancyplain}

\begin{document}
\section*{\LARGE{Tarea 2}}


\begin{enumerate}
  \item Sea $D$ una digr\'afica de orden $n$.   Demuestre que si $D$ no tiene
    ciclos dirigidos, entonces existe un orden total, $v_1, \dots, v_n$ de
    $V_D$, tal que siempre que $(v_i, v_j)$ sea una flecha de $D$, se tiene que
    $i < j$.

  \item Demuestre que si $G$ tiene di\'ametro mayor que $3$, entonces
    $\overline{G}$ tiene di\'ametro menor que $3$.

  \item Sea $G$ una gr\'afica conexa.   Demuestre que si $G$ no es completa,
    entonces contiente a $P_3$ como subgr\'afica inducida.

  \item Demuestre que cualesquiera dos trayectorias de longitud m\'axima en una
    gr\'afica conexa tienen un vértice en común.

  \item Caracterice a las gr\'aficas $k$-regulares para $k \in \{ 0, 1, 2 \}$.

  \item Demuestre que si $|E| \ge |V|$, entonces $G$ contiene un ciclo.
\end{enumerate}

\section*{Puntos extra}

\begin{enumerate}
  \item Sea $G$ una gr\'afica.   Demuestre que $G$ es $k$-partita completa si y
    s\'olo si no contiene a $K_{k+1}$ ni a $\overline{P_3}$ como subgr\'aficas
    inducidas.

  \item Demuestre que si $G$ es una gr\'afica con $|V| \ge 4$ y $|E| > n^2/4$,
    entonces $G$ contiene un ciclo impar.

  \item Sea $d = (d_1, \dots, d_n)$ una sucesi\'on no creciente de enteros no
    negativos.   Sea $d' = (d_2-1, \dots, d_{d_1+1}-1, d_{d_1+2}, \dots, d_n)$.
    \begin{enumerate}
      \item Demuestre que $d$ es gr\'afica si y s\'olo si $d'$ es gr\'afica.

      \item Usando el primer inciso, describa un algoritmo que acepte como
        entrada una sucesi\'on no creciente de enteros no negativos $d$ y
        devuelva una gr\'afica simple con sucesi\'on de grados $d$, un
        certificado de que $d$ no es gr\'afica.
    \end{enumerate}
\end{enumerate}

\end{document}
