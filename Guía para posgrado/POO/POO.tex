\documentclass[12pt]{article}
\usepackage[utf8]{inputenc}
\usepackage[spanish]{babel}
\usepackage{amsmath, amssymb}
\usepackage{listings}
\usepackage{xcolor}
\usepackage{graphicx}
\usepackage{fancyhdr}
\usepackage{titlesec}
\usepackage[top=1cm, bottom=2.5cm, left=2.5cm, right=2.5cm]{geometry} % Reducción del margen superior

% Encabezado bonito
\pagestyle{fancy}
\fancyhf{}
\renewcommand{\headrulewidth}{2pt} % eliminar línea de encabezado
\setlength{\headheight}{30pt} % Ajusta la altura del encabezado
\setlength{\headsep}{1cm}   % Ajusta la separación entre encabezado y texto
\fancyhead[L]{Programación Orientada a Objetos} 
\fancyhead[R]{Java}
\cfoot{\thepage}

% Estilo para títulos
\titleformat{\section}{\large\bfseries\sffamily}{\thesection}{1em}{}

% Configuración de listings para Java
\definecolor{codegray}{gray}{0.95}
\definecolor{javared}{rgb}{0.6,0,0} 
\definecolor{javagreen}{rgb}{0.25,0.5,0.35}
\definecolor{javapurple}{rgb}{0.5,0,0.5}
\definecolor{javadocblue}{rgb}{0.25,0.35,0.75}

\lstdefinelanguage{Java}{
  keywords={abstract, break, case, catch, class, const, continue, default, do, else,
    extends, final, finally, for, if, implements, import, instanceof, interface,
    native, new, null, package, private, protected, public, return, static,
    strictfp, super, switch, synchronized, this, throw, throws, transient, try,
    void, volatile, while},
  keywordstyle=\color{javared}\bfseries,
  commentstyle=\color{javagreen},
  stringstyle=\color{javapurple},
  morecomment=[s][\color{javadocblue}]{/**}{*/},
  morecomment=[l][\color{javagreen}]{//},
  backgroundcolor=\color{codegray},
  showstringspaces=false,
  basicstyle=\ttfamily\small,
  tabsize=2,
  captionpos=b
}
\renewcommand{\lstlistingname}{Ejemplo}

\title{\Huge Programación Orientada a Objetos en Java}
\author{José Ángel Olmedo Guevara}
\date{}


\begin{document}

\maketitle

\section*{Introducción}
La Programación Orientada a Objetos (POO) es un paradigma de programación que organiza el código en unidades llamadas ``objetos'', que contienen atributos (variables de instancia), identidad (nombre), comportamientos(métodos).

\section{Abstracción}
La abstracción consiste en representar entidades del mundo real con clases, mostrando solo la información esencial e ignorando los detalles innecesarios.


\begin{lstlisting}[language=Java, caption={Ejemplo de abstracción}]
abstract class Animal {
    private String nombre;
    private int edad;

    public Animal(String nombre, int edad) {
        this.nombre = nombre;
        this.edad = edad;
    }

    public String getNombre() {
        return nombre;
    }

    public int getEdad() {
        return edad;
    }

    public abstract void hacerSonido();
}
\end{lstlisting}

\newpage
\section{Encapsulamiento}
El encapsulamiento protege los datos del objeto, ocultando los atributos y exponiéndolos mediante métodos públicos (getters y setters).

\begin{lstlisting}[language=Java, caption={Encapsulamiento con atributos privados}]
public class CuentaBancaria {
    private double saldo;

    public CuentaBancaria(double saldoInicial) {
        saldo = saldoInicial;
    }

    public void depositar(double cantidad) {
        if (cantidad > 0) saldo += cantidad;
    }

    public void retirar(double cantidad) {
        if (cantidad > 0 && cantidad <= saldo) saldo -= cantidad;
    }

    public double getSaldo() {
        return saldo;
    }
}
\end{lstlisting}
\newpage
\section{Herencia}
La herencia permite que una clase herede atributos y métodos de otra clase, promoviendo la reutilización del código.

\begin{lstlisting}[language=Java, caption={Ejemplo de herencia}]
class Perro extends Animal {
    public Perro(String nombre, int edad) {
        super(nombre, edad);
    }

    @Override
    public void hacerSonido() {
        System.out.println(getNombre() + " dice: ¡Guau guau!");
    }
}

class Gato extends Animal {
    public Gato(String nombre, int edad) {
        super(nombre, edad);
    }

    @Override
    public void hacerSonido() {
        System.out.println(getNombre() + " dice: ¡Miau!");
    }
}
\end{lstlisting}

\section{Polimorfismo}
El polimorfismo permite que una misma interfaz o clase base se utilice para representar distintos comportamientos en clases derivadas.

\begin{lstlisting}[language=Java, caption={Ejemplo de polimorfismo}]
public class Main {
    public static void main(String[] args) {
        Animal miPerro = new Perro("Fido", 5);
        Animal miGato = new Gato("Michi", 3);

        miPerro.hacerSonido(); // Fido dice: ¡Guau guau!
        miGato.hacerSonido();  // Michi dice: ¡Miau!
    }
}
\end{lstlisting}

\end{document}

